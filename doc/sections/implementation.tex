\section{Umsetzung}
Die Implementierung erfolgt in Rust, einer Programmiersprache, die für ihre Sicherheit, Geschwindigkeit und Nebenläufigkeit bekannt ist.

\subsection{Einlesen der Eingabedaten}
Die Funktion \\
\texttt{read\_data(fixed\_path: String) -> (i32, i32, Vec<i32>)} \\
ist verantwortlich für das Einlesen der Eingabedaten. \\
Sie erwartet entweder den Pfad zu einer Datei als Argument oder fordert den Benutzer zur Eingabe auf. \\
Der Dateiinhalt wird mit Hilfe der \texttt{BufReader}- und \texttt{lines()}-Methoden verarbeitet. \\
Die Anzahl der Spiele jeder Sorte sowie die Gesamtanzahl der Spiele werden anschließend zurückgegeben.\\

\subsection{Initialisierung der Wundertüten}
Die Funktion\\
\texttt{init\_bags(bag\_count: i32) -> Vec<String>}\\
initialisiert leere Wundertüten als Vektoren von Zeichenketten.\\
Jede Wundertüte wird dabei als leere Zeichenkette repräsentiert.\\

\subsection{Verteilungsalgorithmus}
Die Spiele werden gemäß den Vorgaben gleichmäßig auf die Wundertüten verteilt. \\
In der \texttt{main()}-Funktion wird für jedes Spiel die Anzahl berechnet, die theoretisch auf jede Tüte kommen sollte. \\
Diese Spiele werden dann gleichmäßig auf die Tüten verteilt, und der Rest wird auf ungleich verteilte Tüten aufgeteilt.\\
\\
Die Funktion \\
\texttt{append\_asym(index: usize, game: usize, bag\_count: i32, pbags: \&mut Vec<String>) -> usize} 
spielt eine entscheidende Rolle bei der Verteilung der ungleich verteilten Spiele. \\
Sie fügt der aktuellen Wundertüte das Spiel der aktuellen Sorte hinzu und stellt sicher, 
dass die ungleich verteilten Spiele möglichst gleichmäßig auf die Tüten verteilt werden. \\
Der Index wird verwendet, um die Tüten nacheinander zu durchlaufen. \\
\\
Die Ergebnisse, bestehend aus den gefüllten Wundertüten, werden am Ende ausgegeben.
\\
\\
Diese Implementierung ermöglicht eine effiziente und korrekte Umsetzung des Verteilalgorithmus in Rust. \\
Der Fokus liegt darauf, die Spiele möglichst gleichmäßig zu verteilen und \\
gleichzeitig die Anzahlunterschiede zwischen den Wundertüten zu minimieren.