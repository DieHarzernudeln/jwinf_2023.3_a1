\section{Lösungsidee}\label{sec:idea}
\begin{flushleft}
Die zentrale Herausforderung besteht darin, die Spiele möglichst gleichmäßig auf die Wundertüten zu verteilen
und dabei die Unterschiede in der Anzahl der Spiele zwischen den Tüten zu minimieren.
\linebreak
\\
Der Verteilalgorithmus arbeitet iterativ, um sicherzustellen, dass die Spiele nach Möglichkeit gleichmäßig auf die Tüten aufgeteilt werden.
\linebreak
\\

Zunächst wird für jede Spieleart berechnet, wie viele Spiele theoretisch auf jede Wundertüte kommen sollten, 
wenn die Verteilung vollständig gleichmäßig wäre. Dabei wird der Divisionsoperator verwendet, um eine erste Schätzung zu erhalten.
\linebreak
\\
Der Rest dieser Division gibt an, wie viele Spiele auf ungleich verteilte Tüten kommen müssen.\\

Um die ungleich verteilten Spiele zu berücksichtigen, wird für jede verbleibende Spieleart iterativ durch die Wundertüten iteriert.\\
Dabei wird die Spieleart der aktuellen Wundertüte mit den ungleich verteilten Spielen aufgefüllt.\\
Dies geschieht, indem der Index auf die aktuelle Tüte und die Spieleart verwendet wird.\\
Der Index wird dabei nach jedem Hinzufügen inkrementiert und bei Erreichen der letzten Tüte auf null zurückgesetzt.\\
Dies gewährleistet eine gleichmäßige Verteilung der ungleich verteilten Spiele auf die Tüten.
\linebreak
\\
Durch diese iterative Verteilung wird erreicht, dass die Spiele möglichst gleichmäßig auf die Wundertüten aufgeteilt werden, 
und die Unterschiede in der Anzahl der Spiele zwischen den Tüten minimal sind.\\
Die Verwendung von Schleifen und mathematischen Operationen ermöglicht eine effiziente Implementierung dieses Verteilalgorithmus in Rust.
\end{flushleft}
